\documentclass[]{article}
\usepackage{lmodern}
\usepackage{setspace}
\setstretch{1.5}
\usepackage{amssymb,amsmath}
\usepackage{ifxetex,ifluatex}
\usepackage{fixltx2e} % provides \textsubscript
\ifnum 0\ifxetex 1\fi\ifluatex 1\fi=0 % if pdftex
  \usepackage[T1]{fontenc}
  \usepackage[utf8]{inputenc}
\else % if luatex or xelatex
  \ifxetex
    \usepackage{mathspec}
  \else
    \usepackage{fontspec}
  \fi
  \defaultfontfeatures{Ligatures=TeX,Scale=MatchLowercase}
\fi
% use upquote if available, for straight quotes in verbatim environments
\IfFileExists{upquote.sty}{\usepackage{upquote}}{}
% use microtype if available
\IfFileExists{microtype.sty}{%
\usepackage{microtype}
\UseMicrotypeSet[protrusion]{basicmath} % disable protrusion for tt fonts
}{}
\usepackage[margin=1in]{geometry}
\usepackage{hyperref}
\hypersetup{unicode=true,
            pdftitle={Efeitos de condições socioeconômicas sobre a nacionalização dos partidos brasileiros},
            pdfauthor={Evertton Lira},
            pdfborder={0 0 0},
            breaklinks=true}
\urlstyle{same}  % don't use monospace font for urls
\usepackage{longtable,booktabs}
\usepackage{graphicx,grffile}
\makeatletter
\def\maxwidth{\ifdim\Gin@nat@width>\linewidth\linewidth\else\Gin@nat@width\fi}
\def\maxheight{\ifdim\Gin@nat@height>\textheight\textheight\else\Gin@nat@height\fi}
\makeatother
% Scale images if necessary, so that they will not overflow the page
% margins by default, and it is still possible to overwrite the defaults
% using explicit options in \includegraphics[width, height, ...]{}
\setkeys{Gin}{width=\maxwidth,height=\maxheight,keepaspectratio}
\IfFileExists{parskip.sty}{%
\usepackage{parskip}
}{% else
\setlength{\parindent}{0pt}
\setlength{\parskip}{6pt plus 2pt minus 1pt}
}
\setlength{\emergencystretch}{3em}  % prevent overfull lines
\providecommand{\tightlist}{%
  \setlength{\itemsep}{0pt}\setlength{\parskip}{0pt}}
\setcounter{secnumdepth}{0}
% Redefines (sub)paragraphs to behave more like sections
\ifx\paragraph\undefined\else
\let\oldparagraph\paragraph
\renewcommand{\paragraph}[1]{\oldparagraph{#1}\mbox{}}
\fi
\ifx\subparagraph\undefined\else
\let\oldsubparagraph\subparagraph
\renewcommand{\subparagraph}[1]{\oldsubparagraph{#1}\mbox{}}
\fi

%%% Use protect on footnotes to avoid problems with footnotes in titles
\let\rmarkdownfootnote\footnote%
\def\footnote{\protect\rmarkdownfootnote}

%%% Change title format to be more compact
\usepackage{titling}

% Create subtitle command for use in maketitle
\newcommand{\subtitle}[1]{
  \posttitle{
    \begin{center}\large#1\end{center}
    }
}

\setlength{\droptitle}{-2em}

  \title{Efeitos de condições socioeconômicas sobre a nacionalização dos partidos
brasileiros}
    \pretitle{\vspace{\droptitle}\centering\huge}
  \posttitle{\par}
    \author{Evertton Lira}
    \preauthor{\centering\large\emph}
  \postauthor{\par}
      \predate{\centering\large\emph}
  \postdate{\par}
    \date{2019}


\begin{document}
\maketitle

\section{1. Introdução}\label{introducao}

As condições socioeconômicas dos estados afetam a nacionalização dos
partidos brasileiros? A literatura sobre partidos políticos no Brasil e
no mundo tem evidenciado que variáveis políticas, como quantidade de
candidatos, políticas públicas, gasto de campanha etc sofrem efeitos de
variáveis econômicas e sociais, como crescimento, inflação, PIB e
índices de desenvolvimento humano.

No Brasil, os estudos sobre regionalização e nacionalização partidária
não têm um consenso sobre se o partidos são mais regionalizados ou
nacionalizados. Se levados em consideração vários aspectos de
nacionalização, veremos que os partidos são mais regionalizados em umas
dimensões e mais nacionalizados em outras.

O uso de nacionalização partidária tem sido alvo de críticas da
literatura, especialmente pela sua forma de mensuração na dimensão
tempo. Para este trabalho, me preocupo em utilizar os dados que são mais
recomendados pela literatura especializada.

Este trabalho é dividido em seis grandes seções: primeiro, faço uma
construção teórica sobre os temas pertinentes à este trabalho, que são
os partidos políticos, nacionalização e variáveis socioeconômicas. Em
segundo, apresento as hipóteses, dados e métodos deste trabalho. Em
terceiro, traço estatísticas descritivas sobre as variáveis. Quarto,
submeto os dados e modelos a testes de pressupostos. Quinto, apresento
resultados dos modelos de regressão linear. Por último, apresento as
conclusões.

\section{2. Os partidos políticos}\label{os-partidos-politicos}

A literatura sobre partidos têm explorado diversas faces e aspectos
destes. Tradicinalmente, entendemos ``partido político'' por
organizações essenciais à manutenção da democracia, já que para todo
cargo eletivo (ao menos no Brasil) é necessário estar filiado a um
partido político. Os partidos, portanto, têm características de
estruturar a competição eleitoral, governar, agregar os interesses da
população e pautar leis (Amaral, 2013).

Os conceitos de \emph{partidos de massas} e \emph{partidos de quadros}
(Duverger, 1970) ainda são úteis para pensar nas estruturas dos
partidos. Enquanto os primeiros têm origem externa aos parlamento e são
fortemente organizados, os últimos têm uma organização interna mais
fluida, formada por políticos de carreira e concentração das decisões
internas por parte de uma elite (esta informação será importante mais à
frente).

Dentro das características dos partidos citadas, a que terá foco neste
trabalho é a que compreende a arena eleitoral. Os partidos são
organizações de homens com um fim em comum que buscam controlar o
aparato do governo através de meios legais (Downs, 1999). A forma como
cada partido estrutura as eleições, as campanhas, as candidaturas etc
passa por estruturas partidárias específicas, já que o acesso a
determinadas instâncias de decisão são controladas por elites mais ou
menos hierarquizadas e oligarquizadas (Guarnieri, 2009).

\subsection{2.1. Partidos políticos no Brasil: legislação e
estrutura}\label{partidos-politicos-no-brasil-legislacao-e-estrutura}

A nacionalização partidária (a ser explanada na próxima seção), objeto
deste estudo, é afetada por diversos fatores que resultam da competição
eleitoral. Um destes é a quantidade de votos recebida que, por sua vez,
é afetada pela quantidade de candidatos que um partido lança. Faz-se
necessário algumas explicações sobre este processo. No Brasil, é a Lei
nº 9.504/2011 estabelece que os partidos podem registrar candidatos até
150\% do número de lugares a preencher.

Lançar ou não lançar um candidato é resultado das decisões que são
tomadas nas instâncias partidárias responsáveis pela seleção e registro
de candidatos. Os partidos têm, de maneira perene, seis instâncias:
órgãos de deliberação; direção e ação partidária; ação parlamentar;
órgãos auxiliares; pesquisa, doutrinação e educação política e; órgãos
de cooperação (Guarnieri, 2009). Os órgãos de deliberação são onde os
candidatos e alianças eleitorais são definidos, além da estratégia de
campanha.

Estratégias eleitorais são um momento importante para os resultados
futuros. A decisão de onde lançar mais ou menos candidatos e como
competir eleitoralmente é uma forma importante de galgar sucesso
eleitoral ou, pelo menos, tentar minar as chances de opositores
(Palfrey, 1984; Cox, 1997). Dados de eleições municipais mostram que
90\% de eleitos para o executivo estavam coligados (Guarnieri, 2009). O
ato de estar coligado é uma estratégia pode multiplicar os votos
recebidos.

\section{3. Nacionalização partidária}\label{nacionalizacao-partidaria}

\subsection{3.1. Conceitos e aplicações}\label{conceitos-e-aplicacoes}

Nacionalização é um conceito que permeia por algumas áreas e com certa
frequência causa alguma confusão conceitual. A literatura trabalha com
três conceitos de nacionalização dos partidos: (1) nacionalização
partidária, (2) nacionalização política e (3) nacionalização das
políticas. O primeiro diz respeito à distribuição das estruturas e dos
votos de um partido, de maneira equilibrada, no território nacional. O
segundo, ao conteúdo político, à agenda de debates num país, num dado
momento. O terceiro e último, à formulação e implementação de políticas
públicas ao longo do território nacional (Jones e Mainwaring, 2003;
Morgenstern, 2005). Para os propósitos deste trabalho, focaremos na
\emph{nacionalização partidária}.

A importância de se observar este processo (que é um \emph{continuum}) é
descrita por Conceição (2018):

\begin{quote}
``Separando esses três processos, podemos perceber que um partido
nacional não necessariamente defenderá apenas políticas nacionais, da
mesma forma que um partido regional não necessariamente defenderá apenas
políticas regionais. Ainda, um partido regional e um partido nacional
podem tanto debater temas políticos nacionais -- nacionalização da
política -- como temas políticos locais -- regionalismo da política. Em
termos práticos, é de se esperar que partidos nacionais e regionais
debatam, em graus variados, temas regionais e nacionais, que defendam,
também em graus variados, políticas regionais e nacionais.'' (p.~80)
\end{quote}

A nacionalização partidária é frequentemente observada dentro de dois
aspectos: estático e dinâmico. A nacionalização estática diz respeito a
distribuição dos votos de um partido (ao se observar a nacionalização do
partido) ou de um sistema (ao se observar um conjunto de partidos
políticos) em um determinado ponto no tempo (uma eleição) ao longo de um
território. A nacionalização dinâmica observa a mesma distribuição de
votos em diversos pontos do tempo (Morgenstern et al., 2009). Quanto
mais homogêneas as variações nos votos entre os distritos, mais
nacionalizado o partido é.

Há um pressuposto normativo forte na literatura de que a evolução das
democracias acompanha a evolução dos partidos políticos, que sairiam de
partidos regionalizados para partidos mais nacionalizados (Conceição,
2018). Entendemos o por quê desse pressuposto: ao se administrar um
país, um partido não deveria incorrer em políticas localizadas ou mesmo
clienteslistas. No entanto, dadas as necessidades específicas de cada
região do país, é difícil operacionalizar de maneira satisfatória a
diferença de uma política de ação localista para uma política de que
atenda demandas específicas.

Da mesma maneira que se argumenta que o cerne da vida política no Brasil
são os estados (Nicolau, 1996), há uma ligação entre a fragmentação
partidária e partidos regionais (Vasselai, 2015). A fragmentação
partidária no nível nacional é fruto, também, da competição política e
alianças feitas no nível subnacional (dado que no Brasil os distritos
eleitorais são os estados). Com isso, a fragmentação partidária poderia
ser exacerbada através de mais partidos regionais, já agremiações
políticas diversas ganhariam força no nível nacional. Todos estes
fatores contribuem para maiores ou menores nivel de nacionalização
partidária.

Não há um consenso sobre qual medida de nacionalização (estática ou
dinâmica) é mais útil. Na verdade, cada uma tem seu valor teórico e
metodológico. Analisar a nacionalização num ponte específico de tempo
(uma eleição) pode ser útil para avaliar resultados futuros de políticas
ou de organização da vida política. Em termos metodológicos, observa-se
a homogeneidade da distribuição dos votos (Clagett, Flanigan and Zingale
(1984). A nacionalização dinâmica, por outro lado, pode nos ajudar a
entender o caminho que as democrcias percorrem.

Ao confrontar as duas perspectivas, a literatura nos mostra que:

\begin{quote}
``a crítica que esses autores apresentaram à perspectiva dinâmica da
nacionalização convenceu diversos pesquisadores a respeito de sua
inadequação para capturar o fenômeno. De acordo com eles, a mensuração
dinâmica, na verdade, apreende a nacionalização dos padrões de
volatilidade eleitoral, uma vez que sua unidade de análise é a variação
nas votações dos partidos. Já a mensuração estática, por se concentrar
na análise da homogeneidade das votações partidárias nos distritos em
determinada eleição, de fato captura a efetiva uniformidade da presença
dos partidos no território nacional. Isso significa que apenas a
mensuração estática é a indicada para o estudo da nacionalização
partidária pelo viés da inserção eleitoral das legendas (Conceição,
2018).''
\end{quote}

Para os fins deste trabalho, no entanto, utilizaremos dados de quatro
eleições. Já que o objetivo aqui é tentar traçar uma relação de
causalidade entre condições socioeconômicas e o grau de nacionalização,
julgo que quanto mais dados a análise obtiver, maior será a capacidade
de explicaçao do fenômeno. Como este trabalho também não se propõe a
medir nacionalização em si, é possível utilizar a perspectiva de
nacionalização dinâmica sem perda de qualidade teórico e metodológica.

Além do componente temporal da análise, é comum haver componentes
estruturais na análise sobre nacionalização. Uma destas é a difusão
territorial da estrutura partidária (Chhibber e Kollman, 2004; Mair,
1987). Como outros trabalhos apontam, há grandes diferenças nas
estruturas organizacionais dos partidos dentro deles mesmos e entre as
unidades subnacionais. Estas diferenças normalmente são frutos da forma
como as estruturas foram criadas e também dos interesses políticos das
elites dos partidos (Guarnieri, 2009; Lira, 2017).

Contemporaneamente, quatro dimensoões são utilizados para mensuração da
nacionalização partidária: (1) organizacional, mensurada a partir da
distribuição das estruturas partidárias no território nacional; (2)
oferta de candidatos, operacionalizada a partir da quantidade e
homogeneidade de candidatos lançados para a Câmara baixa; (3) demanda
eleitoral, contabilizada a partir dos votos recebidos entre os distritos
e; (4) retorno eleitoral, que são as cadeiras conquistadas (Vasselai,
2015).

No caso brasileiro, por exemplo, partidos mais antigos e maiores possuem
estruturas organizacionais mais nacionalizadas (Guarnieri, Peres e
Ricci, 2018). A explicação para este fato é que os partidos mais antigos
e maiores ou já existiam no período de redemocratização e início da
atual República ou herdaram estruturas organizacionais de partidos
pré-existentes. A exemplo: o MDB tem longa trajetória e se constitui
como um dos partidos mais nacionalizados. PT e PSDB, para citar os três
maiores partidos, herdaram algumas estruturas do MDB, já que nasceram de
dissidências deste.

É importante aqui, também, fazer considerações a diferença entre
nacionalização partidária e nacionalização do sistema partidário.
Nacionalização dos partidos não necessariamente relfete nacionalização
do sistema. Partidos podem ser nacionalizados em sistemas não
nacionalizados ou mesmo ter sua intensidade mantida ou elevada em
sistemas subnacionais (Borges, 2015). Neste trabalho, são utilizados
dados para os partidos políticos, portanto, busca-se avaliar a
nacionalização dos partidos.

\subsection{3.2. Nacionalização partidária no
Brasil}\label{nacionalizacao-partidaria-no-brasil}

Parte dos estudos sobre os partidos políticos ou sobre o sistema
partidário brasileiro continha uma abordagem que mostrava um caráter
regional dos partidos políticos (Soares, 2001). Até mesmo os trabalhos
sobre relações entre a política e suas bases demonstrava esse aspecto
localizado da ação política (Ames, 2001). Este fato não é à toa, o
sistema partidário brasileiro teve muitas interrupções (Kinzo, 1993),
seja por meio de golpes de estado (1930 e 1964) ou com o fim de regimes
autoritários (1945 e 1985).

A regionalização dos partidos brasileiros era uma constante em diversos
momentos da história política do país (Soares, 2001; Nicolau, 2012).
Partidos que existiam durante o Império eram considerados regionais e de
\emph{quadros}. A Proclamação da República trouxe o federalismo e o
fortalecimento da agenda de estados, vide São Paulo e Minas Gerais, que
tinham fortes partidos regionais e tinham domínio de oligarquias e
interesses locais para manutenção do poder vigente de governadores e
políticos locais (idem).

Estes fenômenos, somados ao desenho constitucional, são os
frequentemente ``culpados'' pela alta fragmentação partidária e baixo
grau de nacionalização dos partidos políticos brasileiros (Ames, 2001).
O atual sistema partidário, portanto, estaria numa situação incipiente
de baixa nacionalização e alta fragmentação, o que, do ponto de vista
normativo, teria efeitos negativos para a o desenvolvimento e
consolidação da democracia, bem como da eficiência das instituiçõe
políticas.

Quando se observa a literatura sobre nacionalização partidária no
Brasil, a literatura caracteriza os partidos e o sistema partidário
brasileiro como os menos nacionalizados das Américas (Jones e
Mainwaring, 2003). Aumentando o escopo de análise, Vasselai (2015) ao
analisar as dimensões de organização, oferta de candidatos, demanda
eleitoral e desempenho eleitoral mostra que os partidos brasileiros são
mais nacionalizados em umas dimensões e mais regionalizados em outras. A
nacionalização partidária seria maior na dimensão organizacional.

Ao observar padrões de coligações eleitorais, medidos a partir de níveis
de disputa e regiões, Krause (2005) aponta baixa nacionalização dos
partidos. Este fenômeno é facilmente observável na prática dadas as
quantidades de críticas que são veiculadas sobre as inconsistências das
coligações dos partidos quando comparamos a eleição nacional com as
eleições estaduais, e também entre estados. Guarnieri, Peres e Ricci
(2018) mostram que os partidos utilizam a nacionalização ou
regionalização como estratégia política de controle das atividades do
partido no âmbito subnacional.

O efeito de ``presidencialização'' das disputas eleitorais também exerce
influÊncia neste aspecto. A polarização entre PT e PSDB para presidência
estava gerando um efeito de atração de outros partidos para a órbita
destes, o que causava incentivos à nacionalizaçao dos partidos (Borges,
2015). Para este trabalho, consideramos também a influência de
candidaturas ao governo dos estados para operacionalizar a influência
que uma candidatura ao Executivo tem sobre a competição eleitoral.

Além das influências institucionais aqui discutidas, um importante
aspecto da nacionalização dos partidos é a continuidade de um sistema.
No caso brasileiro, como mencionado anteriormente, houveram algumas
descontinuidades nos sistemas políticos e partidários. Com o regime
militar de 64, o Brasil passou a ter um autoritarismo parcialmente
fechado, com eleições acontecendo, mesmo que apenas entre dois partido.
Este cenário de autoritarismo competitivo permitiu que partidos como
ARENA e MDB levassem ``herdeiros'' à democracia de 85. Os partidos
herdeiros, PDS, PFL, PMDS e PSDB herdariam, portanto, parte dos quadros
e estruturas dos partidos anteriores (Conceição, 2018).

\section{3.3. Mensuração}\label{mensuracao}

A variável dependente deste trabalho é o \emph{Party Nationalization
Score (PNS)} proposto por Jones and Mainwaring (2003). Esta é uma das
medidas de nacionalização partidária e a mais adequada a este trabalho
por utilizar dados eleitorais. O \emph{PNS} é comumemnte entendido como
``índice de GINI invertido'' pois seu cálculo é semelhante ao índice de
GINI. Enquanto o GINI trata sobre desigualdade na distribuição de renda,
o PNS é sobre homogeneidade da distribuição de estruturas/votos. Por
isto o nome ``invertido'': enquanto no GINI quanto mais próximo de 1,
mais desigual, o PNS é adaptado para quanto mais próximo de 1 for, mais
homogênea é a distribuição.

Apesar da importância, é necessário destacar certas limitações do
índice, como por exemplo a anulação do efeito da magnitude do distrito
no cálculo. Segundo Boschler (2006, p.~31):

\begin{quote}
O coeficiente de Gini especifica o valor 0 para distribuições
perfeitamente iguais (um partido tem exatamente a mesma proporção de
votos em todas as unidades territoriais) e o valor 1 para distribuições
perfeitamente desiguais (todos os votos do partido estão concentrados em
um único ponto do país). Jones e Mainwaring invertem esta escala para o
seu ``Pontuação da Nacionalização Partidária'' (PNS = 1 - coeficiente de
Gini). Sua pontuação é calculada em um primeiro passo para cada partido
político, e depois calculada em média para todo o sistema partidário. No
entanto, quanto maior o número de unidades territoriais em que um país
está dividido, menor o valor do escore PNS. Assim, se tivermos dados
mais detalhados para um país, seu sistema partidário pareceria ser mais
heterogêneo do que se tomasse apenas unidades maiores como base para o
cálculo. Em consequência, o PNS só pode ser comparado entre países com o
mesmo número de unidades territoriais. Essa é uma restrição
incontornável, uma vez que em alguns países temos dados muito finos e em
outros apenas dados de meia dúzia de unidades. É por isso que proponho
uma normalização do indicador pelo número de unidades territoriais, a
fim de transformar o indicador num formato comparável. Eu uso o número
de 10 unidades como um padrão para a comparação. Eu suponho (e mostro
empiricamente) que o indicador PNS aumenta exponencialmente com o
logaritmo do número de unidades tomadas em conta."
\end{quote}

Quando observados os dados sobre nacionalização dos partidos, temos que,
em média, os partidos mais nacionalizados são PMDB (0,73), PT (0,72) e
PSDB (0,66). Estes dados são para a Câmara dos deputados no período
1998-2014. Ao obsevar os \emph{scores} para as assembleias legislativas
estaduais, temos: PT e PMDB (0,73) e PSDB (0,66) (Conceição, 2018).
Assim como a teoria previa, os partidos herdeiros de outros existentes
no sistema político anterior são os mais nacionalizados.

Conceição (2018) faz uma tipologia de nacionalização dos partidos
brasileiros a partir das classificações dos partidos feitas por
Morgenstern (2005):

\subsubsection{Tabela 1: Classificação da Nacionalização dos partidos
(Conceição,
2018)}\label{tabela-1-classificacao-da-nacionalizacao-dos-partidos-conceicao-2018}

\begin{longtable}[]{@{}llll@{}}
\toprule
\begin{minipage}[b]{0.09\columnwidth}\raggedright\strut
\textbf{Tipos}\strut
\end{minipage} & \begin{minipage}[b]{0.15\columnwidth}\raggedright\strut
\textbf{1945-1964}\strut
\end{minipage} & \begin{minipage}[b]{0.15\columnwidth}\raggedright\strut
\textbf{1965-1985}\strut
\end{minipage} & \begin{minipage}[b]{0.14\columnwidth}\raggedright\strut
\textbf{1986-2014}\strut
\end{minipage}\tabularnewline
\midrule
\endhead
\begin{minipage}[t]{0.09\columnwidth}\raggedright\strut
Nacionais\strut
\end{minipage} & \begin{minipage}[t]{0.15\columnwidth}\raggedright\strut
PSD\strut
\end{minipage} & \begin{minipage}[t]{0.15\columnwidth}\raggedright\strut
ARENA e MDB\strut
\end{minipage} & \begin{minipage}[t]{0.14\columnwidth}\raggedright\strut
PMDB e PT\strut
\end{minipage}\tabularnewline
\begin{minipage}[t]{0.09\columnwidth}\raggedright\strut
Desnivelados\strut
\end{minipage} & \begin{minipage}[t]{0.15\columnwidth}\raggedright\strut
UDN, PTB e PCB\strut
\end{minipage} & \begin{minipage}[t]{0.15\columnwidth}\raggedright\strut
-\strut
\end{minipage} & \begin{minipage}[t]{0.14\columnwidth}\raggedright\strut
DEM, PDT, PP, PR, PSDB, PTB e SD\strut
\end{minipage}\tabularnewline
\begin{minipage}[t]{0.09\columnwidth}\raggedright\strut
Em fluxo\strut
\end{minipage} & \begin{minipage}[t]{0.15\columnwidth}\raggedright\strut
PSP\strut
\end{minipage} & \begin{minipage}[t]{0.15\columnwidth}\raggedright\strut
-\strut
\end{minipage} & \begin{minipage}[t]{0.14\columnwidth}\raggedright\strut
PC do B, PHS, PPS, PRB, PROS, PRP, PSB, PSC, PSD, PSOL, PSTU, PTC e
PV\strut
\end{minipage}\tabularnewline
\begin{minipage}[t]{0.09\columnwidth}\raggedright\strut
Localizados\strut
\end{minipage} & \begin{minipage}[t]{0.15\columnwidth}\raggedright\strut
PTN, PR, PSB, PPS, PDC, PRP, PRProg, PL, PRD, PAN, PRT, POT, PST e
MTR\strut
\end{minipage} & \begin{minipage}[t]{0.15\columnwidth}\raggedright\strut
-\strut
\end{minipage} & \begin{minipage}[t]{0.14\columnwidth}\raggedright\strut
PAN, PCB, PCO, PEN, PGT, PMN, PPL, PRONA, PRTB, PSDC, PSL, PST, PT do B
e PTN\strut
\end{minipage}\tabularnewline
\bottomrule
\end{longtable}

\section{4. Efeitos de condições socioeconômicas sobre variáveis
políticas}\label{efeitos-de-condicoes-socioeconomicas-sobre-variaveis-politicas}

Nas seções anteriores mencionei que os primeiros partidos políticos e
também os partidos pós-Era Vargas já tinham um caráter regional. A
tabela 1 sumariza bem estas informações. Para entender melhor a
complexidade das relações econômicas e seus resultados na política, cabe
aqui ressaltar algumas constatações da literatura.

Com o fortalecimento das prerrogativas dos estados e, por consequência,
o fortalecimento dos partidos estaduais (em especial nos estados de São
Paulo e Minas Gerais), observa-se com maior força a regionalização dos
partidos políticos brasileiros. Este processo foi interrompido com o
regime do Estado Novo, que extinguia os partidos políticos e
centralizava o poder nas mãos do presidente.

A República de 1946, por sua vez, restauraria estas bases regionais e
socioeconômicas. Quatro eram os grandes partidos que representavam a
lógica dicotômica regionalismo x nacionalização da época: PSD, UDN, PTB
e PCB. O PSD (partido mais nacionalizado) era formado por interventores
do governo Vargas (Soares, 2001). O grau de nacionalização do partido
era mais elevado, com uma média de 0,75 (Conceição, 2018), sendo o maior
destes grandes partidos. Pode-se atribuir este fato ao controle que os
governadores tinham sobre as estruturas locais dos partidos (Soares,
2001).

A UDN era formada por políticos anti-getulistas, dado que perderam poder
quando Vargas assumiu o poder em 1930 (Soares, 2001). O partido tinha
uma nível de nacionalização médio de 0,66 (Conceição, 2018). O quadro da
UDN não tinha uma posição programática, mas sim servia mais como
oposição ao governo vigente. Suas bases socioeconômicas eram
majoritariamente áreas rurais (onde o PSD também tinha espaço), mas
também conseguiu ganhar votos consideráveis nas regiões metropolitanas,
porém nas classes mais altas.

O PTB, por sua vez, era um partido que conseguia mais votos e adeptos
entre trabalhadores industriais, indo à contramão dos partidos
mencionados anteriormente. O partido tinha um índice de nacionalização
médio de 0,64 e possuía uma estrutura muito hierárquica (Soares, 2001).
Estes partidos, portanto, conseguem retratar um pouco a lógica de
competição dos partidos políticos desta época.

A Lei nº 9.096/1995, que dispõe sobre os partidos políticos determina
que os partidos não podem ter caráter regional:

\begin{quote}
Art. 7º, § 1º Só é admitido o registro do estatuto de partido político
que tenha caráter nacional, considerando-se como tal aquele que
comprove, no período de dois anos, o apoiamento de eleitores não
filiados a partido político, correspondente a, pelo menos, 0,5\% (cinco
décimos por cento) dos votos dados na última eleição geral para a Câmara
dos Deputados, não computados os votos em branco e os nulos,
distribuídos por um terço, ou mais, dos estados, com um mínimo de 0,1\%
(um décimo por cento) do eleitorado que haja votado em cada um deles.
\end{quote}

Na prática, os partidos podem ter estratégias políticas diferentes, como
ter mais ou menos diretórios/comissões provisórias em certos
estados/regiões, ou mesmo competir em regiões que lhe sejam
eleitoralmente mais atrativas. O caminho inverso também é possível: que
partidos sejam mais bem votados em determinadas regiões.

A literatura tem mostrado que um partido como PT (no início dos anos
2000) eram mais bem votado em capitais e grandes centros urbanos
(Carvalho, 2003). Esses votos, no entanto, migraram nas eleições de 2006
para as regiões Norte e Nordeste, principalmente entre os municípios com
menor renda per capita, menores índices de desenvolvimento humano, menor
população e também uma taxa de urbanização menor (Maciel e Ventura,
2017). Para os autores (p.~97):

\begin{quote}
``{[}\ldots{}{]} as bases territoriais de apoio do PT no Legislativo
federal apresentam mudanças em suas características socioeconômicas a
partir do ano de 2006. Nos anos em que esteve na oposição o partido foi
mais bem votado em municípios mais urbanizados e com maior renda per
capita. No pleito em que Lula foi reeleito, a votação para a Câmara foi
maior em municípios menos urbanizados e com menor renda per capita, e as
eleições que se seguiram mantiveram essa tendência. {[}\ldots{}{]} As
análises descritivas demonstram que também a partir do ano de 2006 a
votação dos candidatos petistas a deputado federal começou a crescer na
região Nordeste em detrimento das regiões Sul e Sudeste. A eleição de
2014, por sua vez, foi a primeira em que o partido apresentou
porcentagem de votação maior em cidades pequenas, com até 20 mil
habitantes''.
\end{quote}

O PMDB é mais votado em municípios com IDH maior. O PSDB, por sua vez, é
mais votados em municípios com IDH alto e população menor. O DEM é mais
bem votado em municípios pobres do interior (Maciel, 2014). O que esses
dados nos mostram é que os partidos políticos podem ter relações
socioeconômicas bem estabelecidas com suas base mais fortes, o que
provavelmente estaria levando estes partidos a ter mais votos. Lira
(2017) mostra com dados para as assembleias estaduais que à medida que
PIB e IDH aumentam, a quantidade de candidatos lançados pelos partidos
também aumentam.

Há também um padrão na composição da Câmara, segundo Rodrigues (2009,
p.~136):

\begin{quote}
``1. Alta proporção de empresários na direita, menor proporção nocentro
e quase inexistente na esquerda 2. Alta proporção de deputados que
exerceram profissões liberais e intelectuais nos três blocos, embora um
pouco mais elevada no do centro e principalmente no da esquerda. 3.
Forte presença de professores nos partidos de esquerda, e mais fraca
entre os partidos de centro e de direita 4. Muitos funcionários das
altas administrações públicas dos Estados e da União nos partidos de
direita e de centro 5. Proporção de trabalhadores manuais e empregados
não manuais no interior das bancadas dos partidos de esquerda muito mais
elevada do que a encontrada nos partidos de direita e de centro''.
\end{quote}

Como pudemos observar, características socioeconômicas de estados e
municípios são fequentemente observados pela literatura enquanto fator
de importante influência nos resultados eleitorais. Dado que nossa
medida sobre nacionalização partidária é um construto a partir de dados
eleitorais, é natural indagar se as condições socioeconômicas dos
estados vão influenciar na nacionalização dos partidos.

A litertura sobre institucionalização dos partidos e dos sistemas
partidários já identificou que há uma relação positiva entre
desenvolvimento socioeconômico de um país e institucionalização dos
sistemas partidários (Mainwaring, 2018). Quando observados os índices de
nacionalização partidária, percebeu-se que (com dados para os países da
América) os países com maior índices de desenvolvimento tinham maiores
índices de institucionalização partidária, qualidade e longevidade da
democracia.

\section{5. Hipóteses e argumentos}\label{hipoteses-e-argumentos}

Dada as construções teóricas aqui apresentadas, surge a hipótese:

\begin{quote}
H1: Maiores índices de desenvolvimento dos estados levará a um aumento
no \emph{Party Nationalization Score}
\end{quote}

Utilizando os argumentos teóricos apresentados anteriormente de que
estados com maior PIB e IDH têm trajetórias políticas mais bem
consolidadas e são palco de maior competição partidária, dado que têm
maior quantidade de candidatos lançados.

\begin{quote}
H2: O aumento no \emph{Party Nationalization Score} em relação aos
índices de desenvolvimento é ainda maior entre os partidos de esquerda
\end{quote}

Como parte da literatura argumenta que os partidos de esquerda são mais
programáticos e com maior orientação ideológica que partidos de centro e
de direita (Carvalho, 2003), espera-se que os índices de nacionalização
sejam maiores nestes partidos, levando em consideração a variação a
partir dos índices de desenvolvimento dos estados.

\section{6. Dados e métodos}\label{dados-e-metodos}

Para operacionalização deste trabalho, serão utilizados dados sobre
competição eleitoral, condições socioeconômicas e nacionalização
disponíveis em Lira (2017). Os dados sobre nacionalização foram
retirados do dados de Conceição (2018).

\subsubsection{Tabela 2: Variável dependente, mensuração e
fonte}\label{tabela-2-variavel-dependente-mensuracao-e-fonte}

\begin{longtable}[]{@{}lll@{}}
\toprule
\begin{minipage}[b]{0.19\columnwidth}\raggedright\strut
\textbf{Variável dependente}\strut
\end{minipage} & \begin{minipage}[b]{0.22\columnwidth}\raggedright\strut
\textbf{Mensuração}\strut
\end{minipage} & \begin{minipage}[b]{0.15\columnwidth}\raggedright\strut
\textbf{Fontes}\strut
\end{minipage}\tabularnewline
\midrule
\endhead
\begin{minipage}[t]{0.19\columnwidth}\raggedright\strut
Party Nationalization Score\strut
\end{minipage} & \begin{minipage}[t]{0.22\columnwidth}\raggedright\strut
Média do PNS dos partidos para as assembleias estaduais\strut
\end{minipage} & \begin{minipage}[t]{0.15\columnwidth}\raggedright\strut
Conceição (2018)\strut
\end{minipage}\tabularnewline
\bottomrule
\end{longtable}

\subsubsection{Tabela 3: variáveis independentes, mensuração e
fontes}\label{tabela-3-variaveis-independentes-mensuracao-e-fontes}

\begin{longtable}[]{@{}lll@{}}
\toprule
\begin{minipage}[b]{0.37\columnwidth}\raggedright\strut
\textbf{Variáveis independentes}\strut
\end{minipage} & \begin{minipage}[b]{0.22\columnwidth}\raggedright\strut
\textbf{Mensuração}\strut
\end{minipage} & \begin{minipage}[b]{0.15\columnwidth}\raggedright\strut
\textbf{Fontes}\strut
\end{minipage}\tabularnewline
\midrule
\endhead
\begin{minipage}[t]{0.37\columnwidth}\raggedright\strut
PIB estadual\strut
\end{minipage} & \begin{minipage}[t]{0.22\columnwidth}\raggedright\strut
PIB em R\$1.000\strut
\end{minipage} & \begin{minipage}[t]{0.15\columnwidth}\raggedright\strut
Lira (2017)\strut
\end{minipage}\tabularnewline
\begin{minipage}[t]{0.37\columnwidth}\raggedright\strut
IDH estadual\strut
\end{minipage} & \begin{minipage}[t]{0.22\columnwidth}\raggedright\strut
IDH estadual no ano da eleição\strut
\end{minipage} & \begin{minipage}[t]{0.15\columnwidth}\raggedright\strut
Lira (2017)\strut
\end{minipage}\tabularnewline
\begin{minipage}[t]{0.37\columnwidth}\raggedright\strut
Candidatos a deputado estadual\strut
\end{minipage} & \begin{minipage}[t]{0.22\columnwidth}\raggedright\strut
Quantidade de candidatos sobre magnitude do distrito\strut
\end{minipage} & \begin{minipage}[t]{0.15\columnwidth}\raggedright\strut
Lira (2017)\strut
\end{minipage}\tabularnewline
\begin{minipage}[t]{0.37\columnwidth}\raggedright\strut
Candidato a governador\strut
\end{minipage} & \begin{minipage}[t]{0.22\columnwidth}\raggedright\strut
\emph{Dummy}: 0= partido sem, 1= partido com candidato\strut
\end{minipage} & \begin{minipage}[t]{0.15\columnwidth}\raggedright\strut
Lira (2017)\strut
\end{minipage}\tabularnewline
\begin{minipage}[t]{0.37\columnwidth}\raggedright\strut
\emph{Dummy} esquerda\strut
\end{minipage} & \begin{minipage}[t]{0.22\columnwidth}\raggedright\strut
1= partidos de esquerda, 0= demais ideologias\strut
\end{minipage} & \begin{minipage}[t]{0.15\columnwidth}\raggedright\strut
Lira (2017)\strut
\end{minipage}\tabularnewline
\bottomrule
\end{longtable}

Para obter os resultados, serão utilizados modelos de regressão linear
de mínimos quadrados ordinários com termo interativo.

\section{7. Resultados}\label{resultados}

\subsection{7.1. Estatísticas
descritivas}\label{estatisticas-descritivas}

Uma das principais variáveis deste trabalho é o PIB. Ele é utilizado
aqui como uma medida de riqueza dos estados, dentro do que categorizamos
enquanto ``condição socioeconômica'', além do IDH. Fazendo uma análise
descritiva do PIB, obtemos:

\newpage

\subsubsection{Gráfico 1: Distribuição do
PIB}\label{grafico-1-distribuicao-do-pib}

\begin{verbatim}
## Loading required package: carData
\end{verbatim}

\begin{verbatim}
## `stat_bin()` using `bins = 30`. Pick better value with `binwidth`.
\end{verbatim}

\includegraphics{Artigofinal_files/figure-latex/unnamed-chunk-1-1.pdf}
Como percebemos, os estados são muito diferentes em termos de Produto
Interno Bruto. Muitos estados têm PIB pequeno, enquanto poucos estados
têm PIB mais elevado. Enquanto o PIB dos estados de Roraima (2.313),
Acre (2.868) e Amapá (3.292) são os mais pobres, estados como Minas
Gerais (516.634), Rio de Janeiro (671.077) e São Paulo (1.858.196) são
os mais ricos do país.

Esta distribuição tem uma grande concentração de dados em uma das
extremidades e uma longa cauda. Desta maneira, não estaria se
assemelhando à uma distribuição normal. Sendo assim, para melhorar esta
distribuição, realizamos uma transformação logarítmica:

\newpage

\subsubsection{Gráfico 2: Distribuição logartítmica do
PIB}\label{grafico-2-distribuicao-logartitmica-do-pib}

\begin{verbatim}
## `stat_bin()` using `bins = 30`. Pick better value with `binwidth`.
\end{verbatim}

\includegraphics{Artigofinal_files/figure-latex/unnamed-chunk-2-1.pdf}
Após a transformação percebemos que os dados estão muito mais próximos
de uma distribuição normal, o que será útil para as análises mais à
frente.

As principais variáveis independentes deste trabalho são PIB e IDH. Da
mesma forma, nossa variável independente é o \emph{Party Nationalization
Score}. Vejamos agora a distribuição de PIB e PNS:

\subsubsection{\texorpdfstring{Gráfico 3: \emph{Plot} das variáveis
log(PIB) e
PNS}{Gráfico 3: Plot das variáveis log(PIB) e PNS}}\label{grafico-3-plot-das-variaveis-logpib-e-pns}

\includegraphics{Artigofinal_files/figure-latex/unnamed-chunk-3-1.pdf}
Como podemos observar, a distribuição das duas variáveis se dá por todo
o gráfico, ou seja, para praticamente todos os níveis de log(PIB)
existem índices de nacionalização partidária diferentes. Os \emph{plots}
do gráfico, no entanto, nos mostram uma concentração maior de casos na
região central do gráfico. Para os dados de IDH, obtemos:

\subsubsection{\texorpdfstring{Gráfico 4: \emph{Plot} das variáveis IDH
e
PNS}{Gráfico 4: Plot das variáveis IDH e PNS}}\label{grafico-4-plot-das-variaveis-idh-e-pns}

\includegraphics{Artigofinal_files/figure-latex/unnamed-chunk-4-1.pdf}
De maneira semelhante, IDH se comporta de forma parecida com o PIB. Este
padrão é esperado, dado que PIB e IDH são altamente correlacionados, e
por este motivo são incluídos em diferentes modelos da regressão.

As últimas variáveis independentes deste trabalho são Candidatos ao
Legislativo e Candidatos ao Executivo. Como argumentado anteriormente, a
quantidade de candidatos a deputado afeta a competição e, por
consequência, deve afetar a nacionalização. Em relaçao a segunda
variável, espera-se que a existência de um candidato a governador
aumente a nacionalização do partido, devido ao efeito de
``presidencialização'' das eleições.

Para uma avaliação descritiva dos dados sobre nacionalização e ideologia
do partido, temos:

\newpage

\subsubsection{Gráfico 5: Nacionalização partidária por
ideologia}\label{grafico-5-nacionalizacao-partidaria-por-ideologia}

\includegraphics{Artigofinal_files/figure-latex/unnamed-chunk-5-1.pdf} O
gráfico acima nos traz algumas informações interessantes. O violino em
vermelho são os partido de Centro. Verde, partidos de esquerda. Azul,
partidos de direita. Os partidos de Centro são os mais nacionalizados,
com média de aproximadamente 0,7. Esse dado não reduz a importância da
nossa hipótese interativa, dado que (1) a proporção de partidos de
centro é muito menor em relação às demais ideologias, (2) em termos
teóricos, esses partidos são os maiores herdeiros de estruturas já
existentes.

Os partidos de esquerda têm a maior distribuição na nacionalização, indo
de índices próximos de 0 até pouco mais de 0,8. A média é de
aproximadamente 0,5. A maior densidade da distribuição, no entanto, se
concentra em níveis mais elevados de nacionalização. Por último, os
partidos de direita têm média de nacionalização próxima de 0,4, com a
maior parte da distribuição abaixo desta média.

Veremos agora uma distribuição de candidatos a deputado por ideologia
partidária:

\newpage

\subsubsection{Gráfico 6: Distribuição de candidatos a deputado por
ideologia
partidária}\label{grafico-6-distribuicao-de-candidatos-a-deputado-por-ideologia-partidaria}

\includegraphics{Artigofinal_files/figure-latex/unnamed-chunk-6-1.pdf}
Da mesma maneira, o primeiro violino são partidos de centro; o segundo,
esquerda; o terceiro, direita. Os partidos de Centro são os que lançam
mais candidatos (média 0,7) em comparação às outras ideologias (Esquerda
e Direita com média 0,5). A distribuição maior da Esquerda e da Direita,
no entanto, se concentra próximo de 0.

Anteriormente, mostrei que a legislação (Lei Nº 9.096/1995) permite aos
partidos registrar candidatos em até 150\% do total de vagas. Uma
alteração nos dispositivos, através da Lei Nº 13.165/2015 normatiza que,
em alguns casos, esse percentual pode ser elevado até 200\%:

\begin{quote}
``Art. 10. Cada partido ou coligação poderá registrar candidatos para a
Câmara dos Deputados, a Câmara Legislativa, as Assembleias Legislativas
e as Câmaras Municipais no total de até 150\% (cento e cinquenta por
cento) do número de lugares a preencher, salvo:
\end{quote}

\begin{quote}
I - nas unidades da Federação em que o número de lugares a preencher
para a Câmara dos Deputados não exceder a doze, nas quais cada partido
ou coligação poderá registrar candidatos a Deputado Federal e a Deputado
Estadual ou Distrital no total de até 200\% (duzentos por cento) das
respectivas vagas;
\end{quote}

\begin{quote}
II - nos Municípios de até cem mil eleitores, nos quais cada coligação
poderá registrar candidatos no total de até 200\% (duzentos por cento)
do número de lugares a preencher.
\end{quote}

O que vemos nesse gráfico, no entanto, é a ultrapassagem desse limite. 7
são as ocorrências nas quatro eleições analisadas.

\subsection{7.2. Análise dos pressupostos dos
modelos}\label{analise-dos-pressupostos-dos-modelos}

Quatro serão os modelos utilizados neste trabalho. Para todos a variável
dependente (VD) é o PNS.

\begin{itemize}
\tightlist
\item
  Modelo 1: Candidatos a deputado + candidato a governador + IDH +
  Ideologia
\item
  Modelo 2: Candidatos a deputado + candidato a governador + log(PIB) +
  Ideologia
\item
  Modelo 3: Candidatos a deputado + candidato a governador + dummy
  esquerda*IDH
\item
  Modelo 3: Candidatos a deputado + candidato a governador + dummy
  esquerda*log(PIB)
\end{itemize}

\subsubsection{Pressuposto 1: o modelo de regressão é linear nos
parâmetros}\label{pressuposto-1-o-modelo-de-regressao-e-linear-nos-parametros}

Os modelos não possuem funções exponenciais, portanto, atendem ao
primeiro pressuposto.

\subsubsection{Pressuposto 2: a média dos resíduos é 0 (ou próxima de
0)}\label{pressuposto-2-a-media-dos-residuos-e-0-ou-proxima-de-0}

\begin{verbatim}
## [1] 1.823241e-17
\end{verbatim}

\begin{verbatim}
## [1] -3.709595e-17
\end{verbatim}

\begin{verbatim}
## [1] 1.334718e-17
\end{verbatim}

\begin{verbatim}
## [1] 6.847088e-17
\end{verbatim}

Como podemos perceber, todas as médias dos resíduos são muito próximas
de 0 (elevados a potência negativa). Portanto, o segundo pressuposto
também é atendido.

\newpage

\subsubsection{Pressuposto 3: Homoscedasticidade dos
resíduos}\label{pressuposto-3-homoscedasticidade-dos-residuos}

\textbf{MODELO 1}

\includegraphics{Artigofinal_files/figure-latex/unnamed-chunk-9-1.pdf}

O resíduos do modelo 1 não parecem estar distribuídos aleatoriamente,
não atendendo a este pressuposto.

\newpage

\textbf{MODELO 2}

\includegraphics{Artigofinal_files/figure-latex/unnamed-chunk-10-1.pdf}

Os resíduos do modelo 2 também não têm variância constante.

\newpage

\textbf{MODELO 3}

\includegraphics{Artigofinal_files/figure-latex/unnamed-chunk-11-1.pdf}

Os resíduos também não têm variância constante.

\newpage

\textbf{MODELO 4}

\includegraphics{Artigofinal_files/figure-latex/unnamed-chunk-12-1.pdf}

Os resíduos também não têm variância constante. Uma solução para este
problema é transformar nossa variável dependente utilizando a
transformação de Box-Cox.

\includegraphics{Artigofinal_files/figure-latex/unnamed-chunk-15-1.pdf}
\includegraphics{Artigofinal_files/figure-latex/unnamed-chunk-15-2.pdf}
\includegraphics{Artigofinal_files/figure-latex/unnamed-chunk-15-3.pdf}
\includegraphics{Artigofinal_files/figure-latex/unnamed-chunk-15-4.pdf}

Como percebemos, não houve mudança significativa nos resíduos. Foram
testados modelos lineares generalizados, mas também sem sucesso.
Portanto, o pressuposto 3 não foi atendido.

\subsubsection{Pressuposto 4:
normalidade}\label{pressuposto-4-normalidade}

Como observado nos gráficos de teste de normalidade Q-Q apresentados
anteriormente, os pontos estão bem ajustados à linha, configurando
normalidade dos dados.

\subsubsection{Pressuposto 5: resíduos não são
autocorrelacionados}\label{pressuposto-5-residuos-nao-sao-autocorrelacionados}

\includegraphics{Artigofinal_files/figure-latex/unnamed-chunk-16-1.pdf}
\includegraphics{Artigofinal_files/figure-latex/unnamed-chunk-16-2.pdf}

Quando os resíduos não estão autocorrelacionados, a segunda linha
vertical cai para 0 ou próximo de 0. Todos os modelos atendem a este
pressuposto.

\subsubsection{Pressuposto 6: o número de observações é maior que o
número de
variáveis}\label{pressuposto-6-o-numero-de-observacoes-e-maior-que-o-numero-de-variaveis}

Para cada variável no modelo, o banco de dados apresenta 2.908
observações.

\subsubsection{Pressuposto 7: o modelo está bem
especificado}\label{pressuposto-7-o-modelo-esta-bem-especificado}

O modelo passou por diversos testes, demonstrando estar bem especificado
dentro dos parâmetros.

\subsubsection{Pressuposto 8: sem multicolinearidade
perfeita}\label{pressuposto-8-sem-multicolinearidade-perfeita}

\begin{verbatim}
##              GVIF Df GVIF^(1/(2*Df))
## CANDLEG  1.092323  1        1.045142
## CANDEXEC 1.118271  1        1.057483
## IDH      1.052725  1        1.026024
## IDEOL    1.130425  2        1.031123
\end{verbatim}

\begin{verbatim}
##              GVIF Df GVIF^(1/(2*Df))
## CANDLEG  1.046390  1        1.022932
## CANDEXEC 1.117723  1        1.057224
## log(PIB) 1.010303  1        1.005138
## IDEOL    1.130378  2        1.031112
\end{verbatim}

\begin{verbatim}
##      CANDLEG     CANDEXEC     dummyesq          IDH dummyesq:IDH 
##     1.069950     1.054720    54.578095     1.726815    55.304557
\end{verbatim}

\begin{verbatim}
##           CANDLEG          CANDEXEC          dummyesq          log(PIB) 
##          1.025461          1.052339         62.871545          1.662497 
## dummyesq:log(PIB) 
##         63.779142
\end{verbatim}

Considera-se 4 o ponto de corte para estabelecer se as variáveis têm
multicolineariedade. O que estiver abaixo de 4, atende ao pressuposto.
Como observado, todas as variáveis estão muito abaixo de 4. As exceções
são os termos interativos incluídos. No entanto, faz sentido esse
resultado já que há uma multiplicação entre elas. Conlcuo, poranto, que
os modelos atendem a praticamente todos os pressupostos.

\subsection{7.3. Regressões}\label{regressoes}

\subsubsection{Modelo 1}\label{modelo-1}

\includegraphics{Artigofinal_files/figure-latex/unnamed-chunk-19-1.pdf}

\begin{longtable}[]{@{}lll@{}}
\toprule
\textbf{Residual standar error} & \textbf{R-squared} & \textbf{Adjusted
R-squared}\tabularnewline
\midrule
\endhead
0.1392 & 0.2969 & 0.2957\tabularnewline
\bottomrule
\end{longtable}

No modelo 1, um aumento na variação de Candidatos ao Legislativo aumenta
a variação no PNS, assim como esperado. Candidato ao Executivo, por sua
vez, têm coeficiente negativo. Um aumento de uma unidade no IDH aumenta
em 0.172 o PNS, mostrando que há uma relação positiva entre condições
socioeoconômicas e nacionalização. Ideologia teve um impacto negativo,
mais forte ainda nos partidos de Direita. Com exceção da variável
CANDEXEC, todos os coeficientes são significante a nível 0.

\newpage

\subsubsection{Modelo 2}\label{modelo-2}

\includegraphics{Artigofinal_files/figure-latex/unnamed-chunk-20-1.pdf}

\begin{longtable}[]{@{}lll@{}}
\toprule
\textbf{Residual standar error} & \textbf{R-squared} & \textbf{Adjusted
R-squared}\tabularnewline
\midrule
\endhead
0.1399 & 0.2895 & 0.2883\tabularnewline
\bottomrule
\end{longtable}

O modelo 2 tem resultados semelhantes ao modelo 1. O efeito da variável
Candidatos ao Legislativo é um pouco maior e ainda significativo.
Candidato ao Executivo se comporta da mesma forma, bem como as variáveis
de ideologia. Os coeficientes são muito parecidos. A variável de maior
interesse, log(PIB), teve efeito positivo, porém muito fraco. Um aumento
de 1\% na variaçao do PIB resulta num aumento de 0.0028\% na variação do
PNS. CANDEXEC e log(PIB) não são estatisticamente significativos, ao
passo que as outras variáveis continuam significativas a nível 0.

\newpage

\subsubsection{Modelo 3}\label{modelo-3}

\includegraphics{Artigofinal_files/figure-latex/unnamed-chunk-21-1.pdf}

\begin{longtable}[]{@{}lll@{}}
\toprule
\textbf{Residual standar error} & \textbf{R-squared} & \textbf{Adjusted
R-squared}\tabularnewline
\midrule
\endhead
0.1399 & 0.2895 & 0.2883\tabularnewline
\bottomrule
\end{longtable}

Neste modelo, Candidatos ao Legislativo é positiva. Um aumento de uma
unidade nesta variável gera um aumento de 0.122 no PNS. Aqui,
diferentemente dos outros modelos, Candidatos ao Executivo aparece
positivo e significativo. Quando há um candidato, o PNS aumenta em 0.03.
A \emph{dummy} esquerda tem maior coeficiente (0.26). Contrariando a
expectativa, a interação IDH e Esquerda apresentou um coeficiente
significativo e negativo. Um aumento de uma unidade nesta interação está
gerando uma redução de 0.31 na nacionalização dos partidos.

\newpage

\subsubsection{Modelo 4}\label{modelo-4}

\includegraphics{Artigofinal_files/figure-latex/unnamed-chunk-22-1.pdf}

\begin{longtable}[]{@{}lll@{}}
\toprule
\textbf{Residual standar error} & \textbf{R-squared} & \textbf{Adjusted
R-squared}\tabularnewline
\midrule
\endhead
0.1534 & 0.1461 & 0.1446\tabularnewline
\bottomrule
\end{longtable}

Este é o único modelo em que todas as variáveis são estatisticamente
significativas. Candidatos ao Legislativo, Candidatos ao Executivo e
log(PIB) continuam tendo coeficientes positivos. Um auemnto de 1\% no
PIB representa um aumento de 0.008 na nacionalização do partido. O
efeito de CANDLEG (0.128) é maior que CANDEXEC (0.028). Da mesma
maneira, a \emph{dummy} esquerda tem maior coeficiente, gerando um
aumento de 0.254 quando se aumenta a independente em uma unidade. O
termo interativo, quando aumenta uma unidade, representa uma redução da
nacionalização do partido em 0.017.

\section{8. Conclusões}\label{conclusoes}

Este trabalho apresentou uma discussão sobre efeitos de condições
socioeconômicas sobre a nacionalização dos partidos brasileiros.
Teoricamente, apresentamos que a literatura sobre partidos políticos vêm
estudando os impactos das organizações e estratégias partitdárias sobre
outras variáveis políticas, como a competição eleitoral e a difusão das
estruturas organizacionais, por exemplo.

Além destes fatores, os resultados dos testes empíricos apresentados
aqui mostraram que à medida que aumentam as condições socioeconômicas
dos estados (PIB e IDH), a nacionalização dos partidos também aumenta,
indo ao encontro da nossa principal hipótese. Os dados e modelos
apresentados passaram por vários testes, tendo resultados positivos em
praticamente todos. Os modelos continuaram heteroscedásticos apesar de
todas as recomendações ter sido seguidas.

Este artigo contêm algumas limitações, a começar pelo não cumprimento do
pressuposto da homoscedasticidade. Os dados sobre PNS representam uma
média por partido, por eleição para as assembleias. É preciso levar em
consideração também as estruturas partárias (diretórios e comissões
provisórias) para um estudo mais aprofundado sobre o tema da
nacionalização partidária.

\newpage

\section{9. Referências
bibliográficas}\label{referencias-bibliograficas}

AMARAL, Oswaldo. (2013). O que sabemos sobre a organização dos partidos
políticos: uma avaliação de 100 anos de literatura. Revista Debates, v.
7, n. 2, pp.~11-32.

AMES, Barry. (2001). Os entraves da democracia no Brasil. Rio de
Janeiro, Editora FGV.

BORGES, André. Nacionalização partidária e estratégias eleitorais no
presidencialismo de coalizão. DADOS, vol.58, nº 3, pp.~651- 688. 2015.

BRASIL. Lei nº 9.096/1995. Disponível em:
\url{http://www.planalto.gov.br/ccivil_03/LEIS/L9096.htm}

BRASIL. Lei nº 13.165/2015. Disponível em:
\url{http://www.planalto.gov.br/ccivil_03/_Ato2015-2018/2015/Lei/L13165.htm}

CARVALHO, Nelson Rojas. (2003). E no início eram as bases: geografia
política do voto e comportamento legislativo no Brasil. Rio de Janeiro:
Revan.

CHHIBBER, Pradeep; KOLLMAN, Ken. The Formation of National Party
Systems: Federalism and party competition in Canada, Great Britain,
India, and the United States. Princeton: Princeton University Press,
2004.

CLAGGETT, W; FLANIGAN, W; ZINGALE, N. Nationalization of the American
Electorate. The American Political Science Review, vol.~78, nº1,
pp.~77--91, 1984.

CONCEIÇÃO, Silva (2018). Nacionalização partidária em marcha: processo
de distanciamento dos partidos brasileiros da regionalização
(1945-2014). Tese de doutorado, UFRGS.

COX, Gary. (1997). Electoral rules and electoral coordination. Annual
Review of Political Science, v. 2, pp.~145-161.

DOWNS, Anthony. (1999). Uma teoria econômica da democracia. São Paulo:
EDUSP.

DUVERGER, Maurice. (1970). Os Partidos Políticos. Brasília, Ed. UnB.

GUARNIERI, Fernando. (2009). A força dos ``partidos fracos'' -- um
estudo sobre a organização dos partidos brasileiros e seu impacto na
coordenação eleitoral. Tese de doutorado, USP.

JONES, Mark; MAINWARING, Scott. The Nationalization of parties and party
systems: An empirical measure and an application to Americas. Kellogg
Institute Working Paper. Vol. 1, Nº 304, p.~01-30. 2003.

KINZO, M. D. (1993). Radiografia do Quadro Partidário Brasileiro. Rio de
Janeiro: Konrad-Adenauer Stiftung.

KRAUSE, S. (2005). ``Uma Análise Comparativa das Estratégias Eleitorais
nas Eleições Majoritárias de 1994, 1998 e 2002: Coligações Eleitorais
versus Nacionalização dos Partidos e do Sistema Partidário Brasileiro'';
In: KRAUSE, S. e SCHMITT, R. (Orgs.), Partidos e Coligações Eleitorais
no Brasil. São Paulo: Unesp/KAS.

LIRA, Evertton. (2017). Efeitos de ideologia partidária e bases
socioeconômicas sobre a competição eleitoral dos partidos políticos nos
estados (2002-2014). Trabalho de Conclusão de Curso, UFPE.

MACIEL, Natalia. (2014). Padrões espaciais de voto, bases sociais e
políticas dos deputados federais brasileiros: um estudo comparado entre
PT, PSDB, PMDB e PFL/DEM. 38º Encontro Anual da Anpocs.

MACIEL, Natalia; VENTURA, Tiago. (2017). O Partido dos Trabalhadores na
Câmara dos Deputados: a evolução das bases socioeconômicas e
territoriais (1994-2014). Opinião Pública, vol.~23, nº 1, p.~96-125.

MAINWARING, S., Ed. (2018). Party Systems in Latin America:
Institutionalization, Decay, and Collapse. Cambridge, Cambridge
UniversityPress

MORGENSTERN, Scott; SWINDLE, Stephen. Are politics local? An analysis of
voting patterns in 23 democracies. Comparative Political Studies. Vol.
38, Nº 2, p.~143-170. 2005.

MORGENSTERN, Scott; SWINDLE, Stephen; CASTAGNOLA, Andrea. Party
Nationalization and Institutions. The Journal of Politics. Vol. 71, Nº
4, p.~1322-1341. 2009.

NICOLAU, Jairo. (1996). Multipartidarismo e democracia: um estudo sobre
o sistema partidário brasileiro (1985-94). Rio de Janeiro, Editora FGV.

NICOLAU, Jairo. Eleições no Brasil: do Império aos dias atuais. Rio de
Janeiro: Editora Zahar, 2012.

PALFREY, Thomas. (1984). Spatial equilibrium with entry. Review of
Economic Studies, pp.~139-156.

SOARES, Gláucio Ary Dillon. (2001). A democracia interrompida. Rio de
Janeiro, Editora FGV.

VASSELAI, Fabricio. Nationalization and localism in electoral systems
and party systems. Thesis in Political Science. São Paulo: University of
São Paulo, 2015.


\end{document}
